%%
%% Automatically generated file from DocOnce source
%% (https://github.com/doconce/doconce/)
%% doconce format latex Project2.do.txt --minted_latex_style=trac --latex_admon=paragraph
%%


%-------------------- begin preamble ----------------------

\documentclass[%
oneside,                 % oneside: electronic viewing, twoside: printing
final,                   % draft: marks overfull hboxes, figures with paths
10pt]{article}

\listfiles               %  print all files needed to compile this document

\usepackage{relsize,makeidx,color,setspace,amsmath,amsfonts,amssymb}
\usepackage[table]{xcolor}
\usepackage{bm,ltablex,microtype}

\usepackage[pdftex]{graphicx}

\usepackage[T1]{fontenc}
%\usepackage[latin1]{inputenc}
\usepackage{ucs}
\usepackage[utf8x]{inputenc}

\usepackage{lmodern}         % Latin Modern fonts derived from Computer Modern

% Hyperlinks in PDF:
\definecolor{linkcolor}{rgb}{0,0,0.4}
\usepackage{hyperref}
\hypersetup{
    breaklinks=true,
    colorlinks=true,
    linkcolor=linkcolor,
    urlcolor=linkcolor,
    citecolor=black,
    filecolor=black,
    %filecolor=blue,
    pdfmenubar=true,
    pdftoolbar=true,
    bookmarksdepth=3   % Uncomment (and tweak) for PDF bookmarks with more levels than the TOC
    }
%\hyperbaseurl{}   % hyperlinks are relative to this root

\setcounter{tocdepth}{2}  % levels in table of contents

% prevent orhpans and widows
\clubpenalty = 10000
\widowpenalty = 10000

% --- end of standard preamble for documents ---


% insert custom LaTeX commands...

\raggedbottom
\makeindex
\usepackage[totoc]{idxlayout}   % for index in the toc
\usepackage[nottoc]{tocbibind}  % for references/bibliography in the toc

%-------------------- end preamble ----------------------

\begin{document}

% matching end for #ifdef PREAMBLE

\newcommand{\exercisesection}[1]{\subsection*{#1}}


% ------------------- main content ----------------------



% ----------------- title -------------------------

\thispagestyle{empty}

\begin{center}
{\LARGE\bf
\begin{spacing}{1.25}
Project 2, deadline February 15, 2022
\end{spacing}
}
\end{center}

% ----------------- author(s) -------------------------

\begin{center}
{\bf Erasmus+ Data Analysis and Machine Learning course, Fall/Winter 2021/2022${}^{}$} \\ [0mm]
\end{center}

\begin{center}
% List of all institutions:
\end{center}
    
% ----------------- end author(s) -------------------------

% --- begin date ---
\begin{center}
Nov 3, 2021
\end{center}
% --- end date ---

\vspace{1cm}


\subsection*{Introduction to project 2}

For project 2, you can propose own data sets that relate to your
research interests or just use existing data sets from say

\begin{itemize}
\item \href{{https://www.kaggle.com/datasets}}{Kaggle}

\item \href{{https://archive.ics.uci.edu/ml/index.php}}{The University of California at Irvine (UCI) with its machine learning repository}

\item "The credit card data set from \href{{https://archive.ics.uci.edu/ml/index.php}}{UCI} is also interesting and links to a recent scientific article. See however below for possible project example. See in particular \href{{https://archive.ics.uci.edu/ml/datasets/default+of+credit+card+clients}}{\nolinkurl{https://archive.ics.uci.edu/ml/datasets/default+of+credit+card+clients}} and the \href{{https://www.sciencedirect.com/science/article/abs/pii/S0957417407006719}}{article by Yeh and Lien}. 

\item \href{{https://www.kaggle.com/pavanraj159/predicting-pulsar-star-in-the-universe/notebook?scriptVersionId=4487650}}{The pulsar classification data set is obtained from Kaggle}, where it was posted by Pavan Raj. The data file is available in the DataFiles folder of this project.

\item Or other data sets you find interesting and relevant.
\end{itemize}

\noindent
The approach to the analysis of these new data sets should follow to a
large extent what you did in project 1. That is: Whether you end up
with a regression or a classification problem, you should employ at
least two of the methods we have discussed among linear regression
(including Ridge and Lasso), Logistic Regression, Neural Networks,
Support Vector Machines (not covered during the lectures) and Decision Trees, Random Forests, Bagging and Boosting. If you
wish to venture into convolutional neural networks or recurrent neural
networks, or extensions of neural networks, feel free to do so.  For
project 2, you should feel free to write your own code or use the
available functionality of Scikit-learn, Tensorflow, etc.

The estimates you used and tested in project 1 should also be
included, that is the R2-score, MSE, accuracy scores, cross-validation and/or bootstrap etc
if these are relevant.  If possible, you should link the data sets
with exisiting research and analyses thereof. Scientific articles
which have used Machine Learning algorithms to analyze the data are
highly welcome. Perhaps you can improve previous analyses and even
publish a new article?

A critical assessment of the methods with
ditto perspectives and recommendations is also something you need to
include.  All in all, the report should follow the same pattern with
abstract, introduction, methods, code, results, conclusions etc as in project 1.

\subsection*{Studying the credit card data set as possible project.}

We include this data set as an example on how one could study new data
sets with the algorithms we have discussed during the lectures, using
either your own codes or the functionality of scikit-learn, tensorflow
or other Python packages.

The data set is presented at the site of \href{{https://archive.ics.uci.edu/ml/index.php}}{UCI}. It is particularly
interesting since it is also analyzed using various ML methods in a recent
scientific article.

The authors apply several ML methods, from nearest neighbors via
logistic regression to neural networks and Bayesian analysis (not
covered in our course). Here follows a set up on how to analyze
these data.

\paragraph{Part a).}
The first part deals with structuring and reading the data, much along the same lines as done in project 1.

\paragraph{Part b).}
Perform a logistic regression analysis and see if you can reproduce
the results of figure 3 of the article mentioned in
\href{{https://archive.ics.uci.edu/ml/datasets/default+of+credit+card+clients}}{\nolinkurl{https://archive.ics.uci.edu/ml/datasets/default+of+credit+card+clients}} of \href{{https://www.sciencedirect.com/science/article/abs/pii/S0957417407006719}}{Yeh and Lien}.

\paragraph{Part c).}
The next step is to use neural networks and the functionality provided
by Tensorflow/Keras or Scikit-learn's MLP method (or you could write
your own code). Compare and discuss again your results with those from
the above article.

\paragraph{Part d).}
The above article does not study random forests, bagging and gradient boosting or support vector
machine algorithms. Try to apply one of these methods to the
credit card data and see if these methods provide a better description
of the data. Can you outperform the authors of the article?

\paragraph{Part e).}
Finally, here you should present a critical assessment of the methods
you have studied and link your results with the existing literature.

\paragraph{The Pulsar data.}
The pulsar classification data set is obtained from
Kaggle, where it was posted by Pavan Raj. It offers an interesting
possible classification problem. In the field of radio astronomy,
pulsars are among the most studied phenomena in nature. But despite
astronomers' long history with pulsars, little is actually known with
certainty. However, much of the uncertainty likely boils down to the
difficulty of confirming pulsar observations. While pulsars radiate
unmistakable radio signals, they are often lost in the sheer number of
radio signals observed by radio telescopes every day. Furthermore, due
to the uniqueness of pulsar radio signals, classifying pulsars in
large data sets of radio observations have historically been very
difficult as human supervision has been a necessity. However, recent
advances in machine learning and data mining has made this task
much simpler by introducing incredibly fast, in comparison to humans
that is, classification methods.

You could repeat many of the steps discussed for the credit card data problem.
The article of \href{{https://arxiv.org/abs/1209.0793}}{Bathes et al} can serve as a reference for your discussions.

\paragraph{Other data sets.}
Alternatively, if you would like to test the various algorithms on other data sets, please feel free to do so.

\subsection*{Introduction to numerical projects}

Here follows a brief recipe and recommendation on how to write a report for each
project.

\begin{itemize}
  \item Give a short description of the nature of the problem and the eventual  numerical methods you have used.

  \item Describe the algorithm you have used and/or developed. Here you may find it convenient to use pseudocoding. In many cases you can describe the algorithm in the program itself.

  \item Include the source code of your program. Comment your program properly.

  \item If possible, try to find analytic solutions, or known limits in order to test your program when developing the code.

  \item Include your results either in figure form or in a table. Remember to        label your results. All tables and figures should have relevant captions        and labels on the axes.

  \item Try to evaluate the reliabilty and numerical stability/precision of your results. If possible, include a qualitative and/or quantitative discussion of the numerical stability, eventual loss of precision etc.

  \item Try to give an interpretation of you results in your answers to  the problems.

  \item Critique: if possible include your comments and reflections about the  exercise, whether you felt you learnt something, ideas for improvements and  other thoughts you've made when solving the exercise. We wish to keep this course at the interactive level and your comments can help us improve it.

  \item Try to establish a practice where you log your work at the  computerlab. You may find such a logbook very handy at later stages in your work, especially when you don't properly remember  what a previous test version  of your program did. Here you could also record  the time spent on solving the exercise, various algorithms you may have tested or other topics which you feel worthy of mentioning.
\end{itemize}

\noindent
\subsection*{Format for electronic delivery of report and programs}

The preferred format for the report is a PDF file. You can also use DOC or postscript formats or as an ipython notebook file.  As programming language we prefer that you choose between C/C++, Fortran2008 or Python. The following prescription should be followed when preparing the report:

\begin{itemize}
  \item Send us by email  \textbf{only} the report file or the link to your GitHub/GitLab or similar repos!  Mske sure it is public or if not, give us access. For the source code file(s) you have developed please provide us with your link to your GitHub/GitLab or similar  domain.  The report file should include all of your discussions and a list of the codes you have developed. 

  \item In your GitHub/GitLab or similar repository, please include a folder which contains selected results. These can be in the form of output from your code for a selected set of runs and input parameters.
\end{itemize}

\noindent
Finally, 
we encourage you to collaborate. Optimal working groups consist of 
2-3 students. You can then hand in a common report. 

\subsection*{Software and needed installations}

If you have Python installed (we recommend Python3) and you feel pretty familiar with installing different packages, 
we recommend that you install the following Python packages via \textbf{pip} as
\begin{enumerate}
\item pip install numpy scipy matplotlib ipython scikit-learn tensorflow sympy pandas pillow
\end{enumerate}

\noindent
For Python3, replace \textbf{pip} with \textbf{pip3}.

See below for a discussion of \textbf{tensorflow} and \textbf{scikit-learn}. 

For OSX users we recommend also, after having installed Xcode, to install \textbf{brew}. Brew allows 
for a seamless installation of additional software via for example
\begin{enumerate}
\item brew install python3
\end{enumerate}

\noindent
For Linux users, with its variety of distributions like for example the widely popular Ubuntu distribution
you can use \textbf{pip} as well and simply install Python as 
\begin{enumerate}
\item sudo apt-get install python3  (or python for python2.7)
\end{enumerate}

\noindent
etc etc. 

If you don't want to install various Python packages with their dependencies separately, we recommend two widely used distrubutions which set up  all relevant dependencies for Python, namely
\begin{enumerate}
\item \href{{https://docs.anaconda.com/}}{Anaconda} Anaconda is an open source distribution of the Python and R programming languages for large-scale data processing, predictive analytics, and scientific computing, that aims to simplify package management and deployment. Package versions are managed by the package management system \textbf{conda}

\item \href{{https://www.enthought.com/product/canopy/}}{Enthought canopy}  is a Python distribution for scientific and analytic computing distribution and analysis environment, available for free and under a commercial license.
\end{enumerate}

\noindent
Popular software packages written in Python for ML are

\begin{itemize}
\item \href{{http://scikit-learn.org/stable/}}{Scikit-learn}, 

\item \href{{https://www.tensorflow.org/}}{Tensorflow},

\item \href{{http://pytorch.org/}}{PyTorch} and 

\item \href{{https://keras.io/}}{Keras}.
\end{itemize}

\noindent
These are all freely available at their respective GitHub sites. They 
encompass communities of developers in the thousands or more. And the number
of code developers and contributors keeps increasing.


% ------------------- end of main content ---------------

\end{document}

